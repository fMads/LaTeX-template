\usepackage[utf8]{inputenc}
\usepackage[T1]{fontenc}
\usepackage[margin=1.0in]{geometry}
\usepackage[danish]{babel}
\usepackage{amsmath, amsfonts, amssymb}
\usepackage{float}
\usepackage{here} % added by Lasse
\usepackage{graphicx}
\usepackage[hidelinks]{hyperref}
\usepackage{pdfpages}
\usepackage{listings}
\usepackage{color}
\usepackage{wrapfig}
\usepackage{mathtools} % added by Lasse
\usepackage{fancyhdr}
\usepackage{lastpage}
\newcommand*{\tabbox}[2][t]{\vspace{0pt}\parbox[#1][3.7\baselineskip]{1cm}{\strut#2\strut}}

\setcounter{tocdepth}{1}
\setcounter{topnumber}{2}
\setcounter{bottomnumber}{2}
\setcounter{totalnumber}{4}

\pagestyle{fancy}
\fancypagestyle{plain}{}
\renewcommand{\sectionmark}[1]{}

\setlength\parindent{0pt}

\renewcommand{\topfraction}{0.85}
\renewcommand{\bottomfraction}{0.85}
\renewcommand{\textfraction}{0.15}
\renewcommand{\floatpagefraction}{0.7}

\newcommand{\textCode}{Kodeudsnit}
\renewcommand{\lstlistingname}{\textCode}
\newcommand{\textTable}{Tabel }

\newcommand{\chapterName}{}		% Indeholder mappenavnet på det nuværende kapitel
\newcommand{\sectionName}{}		% Indeholder navnet på den nuværende sektion
\newcommand{\subsectionName}{}	% Indeholder navnet på det nuværende afsnit
\newcommand{\subsubsectionName}{}	% Indeholder navnet på det nuværende underafsnit
\newcommand{\chap}[2]{			% Tilføjer et kapitel og sætter mappenavnet
	\cleardoublepage
	\chapter{#1}\label{chap:#2}
	\renewcommand{\chapterName}{#2}
	\input{sections/#2}
}
\newcommand{\chapX}[2]{			% Tilføjer et kapitel og sætter mappenavnet
	\cleardoublepage
	\chapter{#1}\label{chap:#2}
	\renewcommand{\chapterName}{#2}
}
\newcommand{\chapIntro}[2]{
	\renewcommand{\chapterName}{intro}
	\cleardoublepage
	\chapter*{#1}\label{chap:#2}
	\thispagestyle{empty}
	\graphicspath{{sections/\chapterName /#2/}}
	\input{sections/\chapterName /#2}	
}
\newcommand{\textFig}{Figur }	% Indsætter teksten foran et reference til et billede
\newcommand{\sect}[2]{			% Indsætter en section i det nuværende kapitel
	\renewcommand{\sectionName}{#2}
	\section{#1}\label{sect:\chapterName :#2}
	\graphicspath{{sections/\chapterName /#2/}}
	\input{sections/\chapterName /#2}
}
\newcommand{\subsect}[2]{
	\renewcommand{\subsectionName}{#2}
	\subsection{#1}\label{subsect:\chapterName :\sectionName :#2}
}
\newcommand{\subsubsect}[2]{
	\renewcommand{\subsubsectionName}{#2}
	\subsubsection{#1}\label{subsubsect:\chapterName :\sectionName :\subsectionName :#2}
}
\newcommand{\fig}[3][]{		% Indsætter et billede ud fra kapitel og afsnit
	\includegraphics[#1]{sections/\chapterName /\sectionName /#2}
	\caption{#3}
	\label{fig:\chapterName :\sectionName :#2}
}
\newcommand{\figX}[2][]{		% Indsætter et billede ud fra kapitel og afsnit
	\includegraphics[#1]{sections/\chapterName /\sectionName /#2}
	\label{fig:\chapterName :\sectionName :#2}
}
% Reference til billede i samme section
\newcommand{\reffig}[1]{\textFig\ref{fig:\chapterName :\sectionName :#1}}
\newcommand{\reffigx}[3]{	% Reference til billede i andet kapitel og sektion
	\textFig\ref{fig:#1:#2:#3}
}
\newcommand{\refchap}[1]{Kapitel \ref{chap:#1} "\nameref{chap:#1}"}
\newcommand{\refsect}[2]{Sektion \ref{sect:#1:#2} "\nameref{sect:#1:#2}"}
\newcommand{\refsubsect}[3]{Afsnit \ref{subsect:#1:#2:#3} "\nameref{subsect:#1:#2:#3}"}
\newcommand{\refsubsubsect}[4]{Underafsnit \ref{subsubsect:#1:#2:#3:#4} "\nameref{subsubsect:#1:#2:#3:#4}"}

\graphicspath{{figs/}}		% Mappenavne til figure

\renewcommand{\headrulewidth}{0pt}

\newcommand{\addHeaderFooter}{
	\renewcommand{\headrulewidth}{0.5pt}
	\renewcommand{\footrulewidth}{0.5pt}
	\fancyhead[LE,RO]{\leftmark}
	\fancyfoot[LE,RO]{Side \thepage\ af \pageref{LastPage}}
	\setcounter{page}{1}
}

\newcommand{\insertListoffigures}{
	\phantomsection
	\addcontentsline{toc}{chapter}{\listfigurename}
	\listoffigures
}
\newcommand{\insertListoftables}{
	\phantomsection
	\addcontentsline{toc}{chapter}{\listtablename}
	\listoftables
}
\newcommand{\insertBibtex}[1]{
	\phantomsection
	\addcontentsline{toc}{chapter}{Litteratur}
	\bibliographystyle{IEEE}
	\bibliography{#1}
}
\newcommand{\insertTableofcontents}{
	\tableofcontents
	\thispagestyle{empty}
	\addtocontents{toc}{\protect\thispagestyle{empty}}
}
\newcommand{\documentBegin}{
	\fancyhf{}
	\title{\reportTitle}
}

\newcommand{\reportTitle}{[[\\reportTitle]]}